%%%%%%%%%%%%%%%%%%%%%%%%%%%%%%%%%%%%%%%%%
% "ModernCV" CV and Cover Letter
% LaTeX Template
% Version 1.1 (9/12/12)
%
% This template has been downloaded from:
% http://www.LaTeXTemplates.com
%
% Original author:
% Xavier Danaux (xdanaux@gmail.com)
%
% License:
% CC BY-NC-SA 3.0 (http://creativecommons.org/licenses/by-nc-sa/3.0/)
%
% Important note:
% This template requires the moderncv.cls and .sty files to be in the same
% directory as this .tex file. These files provide the resume style and themes
% used for structuring the document.
%
%%%%%%%%%%%%%%%%%%%%%%%%%%%%%%%%%%%%%%%%%

%----------------------------------------------------------------------------------------
%	PACKAGES AND OTHER DOCUMENT CONFIGURATIONS
%----------------------------------------------------------------------------------------

\documentclass[11pt,a4paper,roman]{moderncv} % Font sizes: 10, 11, or 12; paper sizes: a4paper, letterpaper, a5paper, legalpaper, executivepaper or landscape; font families: sans or roman

\moderncvstyle{banking} % CV theme - options include: 'casual' (default), 'classic', 'oldstyle' and 'banking'
\moderncvcolor{red} % CV color - options include: 'blue' (default), 'orange', 'green', 'red', 'purple', 'grey' and 'black'

\usepackage{lipsum} % Used for inserting dummy 'Lorem ipsum' text into the template
\usepackage{graphicx}
\usepackage[scale=0.815]{geometry} % Reduce document margins
\usepackage{bibentry}
\newcommand{\texturl}[1]{{\color{rgb}{0.02,0.2,0.8} #1}}
%\usepackage{hyperref}
%\setlength{\hintscolumnwidth}{3cm} % Uncomment to change the width of the dates column
%\setlength{\makecvtitlenamewidth}{10cm} % For the 'classic' style, uncomment to adjust the width of the space allocated to your name


% bibliography with mutiple entries
\usepackage[url=false]{multibib}
\newcites{article,unpublished}{{Peer-reviewed publications},{Preprints on bioRxiv}}
% underline name in bibliography
\usepackage{xstring}

\let\originalbibitem\bibitem
\def\bibitem#1#2\par{%
  \noexpandarg
  \originalbibitem{#1}
  \StrSubstitute{#2}{Dileep Kishore}{\textbf{Dileep Kishore}}\par}

%----------------------------------------------------------------------------------------
%	NAME AND CONTACT INFORMATION SECTION
%----------------------------------------------------------------------------------------

\firstname{Dileep} % Your first name
\familyname{Kishore} % Your last name

% All information in this block is optional, comment out any lines you don't need
\address{Boston University}{Boston, MA 02215, USA}
%\phone{}
%\fax{(000) 111 1113}
\email{dkishore@bu.edu}
% \homepage{https://sriki18.github.io/website/}
\social[linkedin]{dileep-kishore}
\social[github]{dileep-kishore}%{Website} % The first argument is the url for the clickable link, the second argument is the url displayed in the template - this allows special characters to be displayed such as the tilde in this example
% \extrainfo{$5^{th}$ year Int. BS-MS}
%\photo[70pt][0.4pt]{pictures/picture} % The first bracket is the picture height, the second is the thickness of the frame around the picture (0pt for no frame)
%\quote{"A witty and playful quotation" - John Smith}

\begin{document}

\makecvtitle % Print the CV title


\section{Education}

\cventry
{2016--present}
{Ph.D. in Bioinformatics}
{Boston University}{Boston, USA}
{GPA -- 4.00/4.00}{Advisor: Prof. Daniel Segrè}
% {Dissertation: ``Computational study of microbe-microbe interactions and their interplay with the environment''}

\cventry
{2011--2016}
{B.Tech (Honors) and M.Tech (Dual Degree) in Biotechnology}
{Indian Institute of Technology -- Madras}{Chennai, India}
{GPA -- 9.23/10.00}{Advisor: Prof. Karthik Raman}

\section{Professional experience}
\cvitemwithcomment{Graduate Student Researcher}{Boston University}{2016 -- Present}
\cvlistitem{\textbf{Thesis}: ``Computational study of microbe-microbe interactions and their interplay with their environment''}
\cvlistitem{Developed \textit{ReLearn}, a reinforcement learning framework for optimal control of microbial communities in bioreactors}
\cvlistitem{Created \textit{MiCoNE}, a 16S sequencing data analysis pipeline for the inference of co-occurrence networks}
\cvlistitem{Constructed a pipeline to estimate binding affinities of microbial metabolites with the Aryl Hydrocarbon Receptor (AHR)}
\cvlistitem{Formulated a mathematical model of the AHR regulatory pathway to study its implications in cancer progression}
\cvlistitem{Assisted in the development of \textit{MIND}, a database and web interface for the visualization and analysis of microbial interaction networks}
\cvlistitem{Initiated the analysis of metagenomics data from the gut microbiome of ultra centenarians and its inference into co-occurrence networks}
\cvlistitem{Advanced the development of \textit{pipeliner}, a nextflow pipeline for the analysis of bulk and single-cell RNA sequencing data}

\cvitemwithcomment{Software developer}{Open-source software}{2018 -- Present}
\cvlistitem{Developed \textit{cayenne}, a Python package for performing stochastic simulations with a Cython backend}
\cvlistitem{Developed a dashboard for calculating daily carbon footprint using the Julia language}
\cvlistitem{Developed \textit{mace}, a web application to calculate total carbon emissions of sourcing recipe ingredients}

\cvitemwithcomment{Research Engineer Intern}{Biocon Limited}{Summer 2014}
\cvlistitem{Optimized impeller and sparger designs for optimal culture growth in large-scale industrial bioreactors}

\cvitemwithcomment{Undergraduate Student Researcher}{Indian Institute of Technology -- Madras}{2015 -- 2016}
\cvlistitem{\textbf{Thesis}: ``Discovering the design principles of circadian rhythms using GPGPUs''}
\cvlistitem{Developed an algorithm for an unbiased search using GPUs to find network topologies capable of oscillations}


\section{Publications}

\subsection{Peer-reviewed publications}
\cvlistitem{Pacheco, A. R., Pauvert, C., \textbf{Kishore, D.}, & Segrè, D. (2022). Toward FAIR Representations of Microbial Interactions. \textit{mSystems}, 7(5). \url{https://doi.org/10.1128/msystems.00659-22}}
\cvlistitem{Federico, A., Karagiannis, T., Karri, K., \textbf{Kishore, D.}, Koga, Y., Campbell, J. D., & Monti, S. (2018). Pipeliner: A Nextflow-Based Framework for the Definition of Sequencing Data Processing Pipelines. \textit{Frontiers in Genetics}. \url{https://doi.org/10.3389/fgene.2019.00614}}

\subsection{Submitted manuscripts}
\cvlistitem{\textbf{Kishore, D.}, Birzu, G., Hu, Z., DeLisi, C., Korolev, K. S., & Segrè, D. Inferring microbial co-occurrence networks from amplicon data: A systematic evaluation. (\textit{revised manuscript in review in mSystems})}

\subsection{Preprints on BioRxiv}

\cvlistitem{Hu, Z., \textbf{Kishore, D.}, Wang, Y., Birzu, G., DeLisi, C., Korolev, K., & Segrè, D. (2022). A resource for the comparison and integration of heterogeneous microbiome networks. \textit{BioRxiv}, 2022.08.07..07.503059. \url{https://doi.org/10.1101/2022.08.07.503059}}
\cvlistitem{\textbf{Kishore, D.}, & Chandrasekaran, S. (2020). Introducing and benchmarking the accuracy of cayenne: A Python package for stochastic simulations. \textit{BioRxiv}, 2020.10.10.334623. \url{https://doi.org/10.1101/2020.10.10.334623}}
\cvlistitem{\textbf{Kishore, D.}, Birzu, G., Hu, Z., DeLisi, C., Korolev, K. S., & Segrè, D. (2020). Inferring microbial co-occurrence networks from amplicon data: A systematic evaluation. \textit{BioRxiv}, 2020.09.23.309781. \url{https://doi.org/10.1101/2020.09.23.309781} (\textit{revised manuscript in review in mSystems})}

\section{Selected posters and talks}

\cvitem{}{(*presenter)}
\cvlistitem{*Dileep Kishore, Pankaj Mehta, Daniel Segrè. \emph{Using deep-RL to control bioreactors: The ReLearn framework}. Talk. Kilachand fellowship presentation sponsored by the Multicellular Design Program (May 2022).}
\cvlistitem{*Dileep Kishore, Gabriel Birzu, Zhenjun Hu, Charles DeLisi, Kirill S. Korolev, and Daniel Segrè. \emph{Inferring microbial co-occurrence networks from 16S data: A systematic evaluation}. Talk and poster. Intelligent Systems for Molecular Biology 2019 (ISMB 2019).}
\cvlistitem{*Dileep Kishore, Gabriel Birzu, Zhenjun Hu, Charles DeLisi, Kirill S. Korolev, and Daniel Segrè. \emph{MIND: The Microbial Interaction Network Database}. Talk and poster presentation. International Workshop on Bioinformatics and Systems Biology 2018 (IBSB 2018).}
\cvlistitem{*Tanya Karagiannis, *Kritika Karri, *Dileep Kishore, Joshua D. Campbell, and Stefano Monti. \emph{Pipeliner: A flexible high-throughput sequencing data analysis framework}. Poster presentation. Intelligent Systems for Molecular Biology 2017 (ISMB 2017).}
\cvlistitem{*Dileep Kishore and Jennifer Reed. \emph{Wild type flux predictions from RB-TnSeq data}. Poster presentation. Student symposium for Khorana, S. N. Bose and Viterbi-India programs 2015.}

\section{Technical Skills}
\cvitem{Bioinformatics}{16S sequence analysis, metagenomic analysis, bulk and single-cell RNA seq analysis}
\cvitem{Systems Biology}{Metabolic modeling, graph theory, network analysis, kinetic modeling}
\cvitem{Statistics}{Hypothesis testing, linear/logistic regression, generalized linear models}
\cvitem{Machine Learning}{supervised (classification/regression), unsupervised, random forest, XGBoost, PCA, network inference}
\cvitem{Deep Learning}{CNNs, RNNs}
\cvitem{Reinforcement Learning}{off-policy (DQN), on-policy (A2C, PPO)}
\cvitem{Mathematical Modeling}{differential equations, stochastic processes, constraint-based modeling}
\cvitem{Optimization}{global, non-linear, linear programming, combinatorial optimization (network traversal, flow algorithms)}

\section{Computer Skills}
\cvitem{Bioinformatics and pipelines}{nextflow, snakemake, QIIME2, HUMAnN, cobrapy}
\cvitem{Python}{NumPy, pandas, scikit-learn, SciPy, Keras, pytorch, Numba, PyCUDA, dash, dask}
\cvitem{Julia}{JuMP.jl, Plots.jl, DataFrames.jl, Pluto.jl, Distributions.jl}
\cvitem{R}{RSTAN, dplyr, ggplot2, igraph, ggraph}
\cvitem{Web and Database}{HTML/CSS, React, Typescript, Flask, Django, PostgreSQL, MySQL}
\cvitem{Cloud}{Docker, Singularity, AWS, Linode}
\cvitem{Other}{MATLAB, Linux, git, C, C++, \LaTeX}


\section{Teaching and leadership experience}

\subsection{Boston University}

\cventry{2019}{BU Bioinformatics Student Organized Symposium}{Organizing committee}{}{}{
  Helped organize the annual symposium hosted by the Boston University bioinformatics program.
  Responsibilities included contacting leading researchers to coordinate talks at the symposium, introducing the speakers, organizing the poster session, and advertising the event to the broader scientific community in Boston.
}

\cventry{2019}{BF550: Foundations of Programming, Data Analytics, and Machine Learning in Python}{Teaching assistant and recitation instructor}{}{}{
  Assisted in teaching programming and machine learning concepts to graduate students.
}

\cventry{2018--2019}{Bioinformatics Programming Workshops}{Organizing committee and instructor}{}{}{
  Organized a series of workshops that taught basic programming concepts, bioinformatics tools, and tools that help facilitate reproducible research.
}

\cventry{2018--2022}{Bioinformatics Research and Interdisciplinary Training Experience}{Instructor}{}{}{
  Organized a series of workshops to introduce basic bioinformatics research to undergraduate researchers and to provide mentorship.
}

\cventry{2018}{Bioinformatics in the Cloud Workshop}{Organizing committee and instructor}{}{}{
  Organized a series of workshops that taught deployment of large data and complex algorithms to the cloud using the Amazon Web Services platform.
}

\subsection{Indian Institute of Technology -- Madras}

\cventry{2015}{Data Structures and Algorithms for Biology}{Teaching assistant}{}{}{
  Assisted in teaching basic bioinformatics data structures and algorithms to undergraduate students.
}

\cventry{2015}{Computational Systems Biology}{Teaching assistant}{}{}{
  Assisted in teaching computational biology, network biology and synthetic biology to undergraduate and graduate students.
}

\section{Awards}
\cvitemwithcomment{Academic Fellowship}{Kilachand fellowship sponsored by the multicellular design program}{2021}
\cvitemwithcomment{Academic best record}{Indian Institute of Technology -- Madras, Biotechnology}{2016}
% \cvitemwithcomment{99.26th percentile of 460,000 applicants}{Indian Institute of Technology Joint Entrance Exam (IIT-JEE)}{2011}


\end{document}
